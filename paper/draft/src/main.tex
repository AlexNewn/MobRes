%! Author = alexnewnham
%! Date = 17/01/2023

% Preamble
\documentclass[12pt]{article}

% Packages
\usepackage{natbib}
\usepackage{amsmath}

% Document
\begin{document}

    \section{Introduction}\label{sec:introduction}
    Since the new millennium, economic inequalities between regions have been increasing in the European
    Union\citep{iammarino2019regional}.
    This problem does not only concern the European Union, but also other developed countries.
    In the United States, for example, after a century of interstate income convergence, the rate of convergence was
    cut in half in the period 1990-2010\citep{ganong2017has}.
    Across the developed world, discontent has grown in territories affected by economic stagnation, or sometimes
    even decline\citep{rodriguez2020institutions}.
    ``The Geography of Discontent'' refers to this growing geographic dissatisfaction\citep{dijkstra2020geography}.
    In recent years, this sense of discontent has had a political impact in many developed
    countries\citep{organisation2019oecd}.
    The percentage of votes for parties strongly opposed to European integration increased from 10 percent to 18
    percent between 2000 and 2018\citep{dijkstra2020geography}.
    This increase is mainly due to regions affected by economic and industrial decline.
    Within these, there are low levels of education and little employment opportunities\citep{becker2017voted}.
    While the wave of discontent that is affecting many developed countries undoubtedly has a spatial dimension,
    current theoretical models of geographic
    economics struggle to explain and understand it. \\

    To some extent, the situation seems to defy standard economic mechanisms.
    In particular, while mobility is seen as a means of reducing disparities, it seems to be running out of steam in
    several developed countries.
    In France, for example, people are less likely to move and, if they do, they will move to places closer than they
    did 30 years ago.
    ~\cite{hsieh2019housing} and\cite{ganong2017has}, among others,suggest that making it easier for people living on
    the outskirts of large cities to move would provide better opportunities and help reduce
    discontent.
    The functioning of the housing market and housing policies, which affect mobility, play a central role.
    Different arguments are made by different authors.
    They emphasize that low rates of internal migration are due to the lack of opportunities for unskilled workers in
    urban areas.
    In their view,lower housing prices in urban centers would only drain the remaining skilled workers from the
    declining areas.
    This debate highlights the need to better understand which factors influence which agents to migrate.

    The importance of analyzing the microeconomic foundations of internal migration choices goes beyond the debate
    mentioned above.
    The mobility of agents is central to the new geographic economy, which is what distinguishes it from the
    international economy\citep{redding2017quantitative}.
    Recently, the need to understand interregional migration has increased due to more refined spatial models.
    A new generation of flexible models capable of providing clear policy recommendations has emerged.
    Unfortunately, agent mobility is often exogenously assumed to be perfect and costless, or non-existent (e.g.,
    ~\cite{davis2012spatial}).
    In these models, the spatial distribution is more nuanced, with agents responding to a wider variety of factors,
    including environmental amenities\citep{storper2018separate}.
    Environmental amenities represent the positive and freely available elements that locations offer.
    Examples are temperature, access to the sea or local gastronomy.
    Although it is an important step forward to include them, these models currently lack a microeconomic basis for
    migration.
    This is problematic because the weights of environmental amenities in the utility of agents are difficult to
    quantify.
    They are therefore estimated in such a way as to equalize the utilities of agents in the space.
    The consequence is that agents unable to migrate can be mistaken for agents with strong preferences to stay in
    that location\citep{desmet2018geography}.
    In other words, this is tantamount to considering that ``those who remain'' in declining territories do so only for
    love of their land.
    To avoid this,~\cite{desmet2018geography} incorporate mobility costs into their model.
    The development of microeconomic foundations describing internal migration will aim to support spatial models by
    providing quantitative clues about the costs of migration, the factors that influence their decisions and their
    interactions with heterogeneous agents.
    The need to quantify the relative effect of different economic policies on the nature, character, and magnitude
    of internal migration, as well as to identify the factors that influence an individual's decision to move to
    different regions, was already expressed in\cite{todaro1976internal}.


    \section{Methodology}\label{sec:methodology}
    We model migration decisions between different regions for different types of workers.
    Each region differs in its average wage per sector.
    The workers are fully informed on the different wages they are expected to receive in the different regions.
    With this information that make an estimation on the future gains that will be received in the different regions.
    These predictions are the same for all individuals.
    After taking into account the discounted sum of gains, the costs related to each region and the cost of migration
    they decide on where to live in the next period.
    They then update the information the following year and make again a new decision based on the updated wages and
    costs.

    \subsection*{Model}

    We consider individuals \textit{i} working in different sectors \textit{j} and living in different commuting
    zones \textit{z}.
    The utility for each individual can be written as :
    \begin{equation}\label{eq:utility function}
    \begin{aligned}
    U_{i t}(x_{i t}, z_{i t}, z_{i t-1}) =  & \hspace{1mm} \delta \ln w_{i t}(x_{i t}, z_{i t})- \mathbb{I}_{\{z_{t} \neq z_{t-1}\}}\Delta + \epsilon  \\
    \end{aligned}
    \end{equation}

    In each period, they look at the average wages in each region Z for their sector $\psi_{0 z t}$ for $\forall z
    \in Z$.
    With this information they predict their wages until their retirement for each region with the following formula:\\
    \begin{equation}\label{eq:forcasting earning function}
    \begin{aligned}
    \ln w_{i z t+j}=\psi_{0 z t}+\psi_1 x_{i t+j}^{2}+\eta_{i z t+j}
    \end{aligned}
    \end{equation}



    \subsection*{Estimation and identification}
    \subsection*{Model fit}


    \section{Datasets}\label{sec:datasets}


    \section{Results}\label{sec:results}


    \section{Conclusion}\label{sec:conclusion}

    \newpage
    {
        \bibliography{main}
        \bibliographystyle{chicago}
    }
\end{document}
